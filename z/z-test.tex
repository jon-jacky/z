\chapter{Z test page}

This page exercises most of the translations provided by the {\bf Z2HTML} tool.

% LaTeX comments are removed.  Section numbers do not appear.
% Sections, subsections and subsubsections are processed by toc.sed 
% to create the table of contents

\section{Symbols} 

Most Z symbols are represented by GIF's.  Symbols which are not
yet supported appear as untranslated LaTeX commands.

% Here is an example of the LaTeX tabbing.
% Z2HTML does not use xxxx \= \kill lines to control column width.
% Instead use LaTeX space ~ to make columns wider - translates to HTML &nbsp;
% Note use of ~~~ in last line before \end{tabbing} below
% Note use of $...$ to select math font inside table entries
% Note you must use ~ in empty columns - see second line under Sequences

\begin{tabbing}
Logic 	\> $true false \lnot \land \lor \implies \iff \forall \exists \exists_1 \LET \vdash \models$ \\
Sets etc. \> $= \neq \in \notin \emptyset \subseteq \subset \{ \} | @ \lambda \mu  \cap \cup \setminus \bigcup \bigcap \power \cross first second \IF \THEN \ELSE \ldata \rdata$ \\
Relations \> $\rel \mapsto \dom \ran \comp \circ \dres \ndres \rres \nrres \limg \rimg \inv \plus \star \oplus \inrel{R} \id$ \\
Functions \> $\pfun \fun \pinj \inj \bij \psurj \surj \ffun \finj $ \\
Numbers etc. \> $\num \nat \nat_1 + - * \div \mod \leq < \geq > \upto min max \# \finset \finset_1$ \\
Sequences \> $\seq \seq_1 \iseq \langle \rangle \cat \dcat head tail last front rev \inseq \filter \extract \disjoint \partition $ \\ 
~ \> $squash \prefix \suffix$ \\
Bags \> $\bag \lbag \rbag count \bcount \otimes \inbag \subbageq \uplus \uminus items$ \\
Schema calc. \> $\defs [ | ] ; . \lnot \land \lor \theta \semi \pipe \hide \project \pre \lblot \rblot $ \\
Conventions ~~~ \> $? ! ' \Delta \Xi$ \\
\end{tabbing}

\section*{Z paragraphs} % Un-numbered \section* looks the same as \section

\subsection{Inline math}

% In Z2HTML \[ ... \] only works if it's all on one line

\[ ConsoleOp \defs SelectDisplay \lor SelectPatientList \lor SelectFieldList \lor  \ldots \lor IgnoreOthers \]

\subsection{Axiomatic definition}

% labels are ignored
\subsubsection{Inline axiomatic definition and Z} \label{sect:inline-axdef}

\begin{axdef} no\_name: NAME \end{axdef}
\begin{zed} no\_patient == no\_name; no\_field == no\_name \end{zed}

\subsubsection{Multiline Z, axiomatic definition boxes}

\begin{axdef}
	cal\_factor: cal\_const \fun VALUE
\end{axdef}

\begin{zed}
	ACCUMULATION == counter \fun VALUE
\also
 	PRESCRIPTION == prescrip \fun VALUE
\end{zed}

\begin{axdef}	
	Preset: studies \fun (FIELD \pfun PRESCRIPTION) \\
	Prescribed: patients \fun (FIELD \pfun PRESCRIPTION) \\
	Accumulated: patients \fun (FIELD \pfun ACCUMULATION) \\
\where
	\forall s: studies @ no\_field \notin \dom (Preset~s) \\
	\forall p: patients @ no\_field \notin \dom (Prescribed~p) \\
\t9		\land \dom~(Prescribed~p) = \dom~(Accumulated~p) 
\end{axdef}

\subsection{Schema boxes} \label{sect:schema} % For now labels are ignored

\begin{schema}{TherapyControl}
	Session \\
	Field \\
	Intlk \\
	\dots \\
	Console \\
\end{schema}

% The following line uses $...$ to indicate Z indentifiers in running text.

The $CheckOut$ operation has two input parameters, the person $p?$ and
the document $d?$.

\begin{schema}{CheckOut}
	\Delta Documents \\
	p?: PERSON \\
	d?: DOCUMENT
\where
	d? \notin \dom checked\_out \\
	(d?, p?) \in permission \\
	checked\_out' = checked\_out \cup \{(d?, p?)\}
\end{schema}

\subsection*{Generic definition box}

% LaTeX tabular - directions about column justification ignored
% Again use $...$ to select math font for identifiers

\begin{tabular}{l c l}
$Q \comp R$ & - & Relational composition: $Q$ composed with $R$ \\
$R \circ Q$ & - & Backward relational composition, same as $Q \comp R$ \\
\end{tabular}

\begin{gendef}[X,Y,Z]
	\_ \comp \_ : (X \rel Y) \cross (Y \rel Z) \fun (X \rel Z) \\
	\_ \circ \_ : (Y \rel Z) \cross (X \rel Y) \fun (X \rel Z) \\
\where
	\forall Q: X \rel Y; R: Y \rel Z @ \\
\t3		Q \comp R = R \circ Q = \{~ x: X; y: Y; z: Z | x \inrel{Q} y \land y \inrel{R} z @ x \mapsto z ~\}
\end{gendef}

\subsection*{Argue environment}

\begin{argue}
 \exists Editor' @ Insert & Definition of precondition \\
\t1 \iff \exists left', right': TEXT | \dots  @ \dots & Expand schemas \\
\t1 \iff \exists left', right': TEXT @ \dots \land \dots & Restricted $\exists$-quantifier \\
\t1 \iff pr \land \# ((left \cat \langle ch? \rangle) \cat right) \leq maxsize & One-point rule
\t1 \iff pr \land \# left + \# \langle ch? \rangle + \# right \leq maxsize & $\# (s \cat t) = \# s + \#t$ \\
\t1 \iff pr \land \# left + 1 + \# right \leq maxsize & $\# \langle x \rangle = 1$ \\
\t1 \iff pr \land \# left + \# right < maxsize & Arithmetic 
\end{argue}

This completes the calculation.  The precondition of $Insert$ is 
$ ch? \in printing \land \# left + \# right < maxsize$.  When the input is a
printing character, the number of characters to
the left and right of the cursor must be less than the buffer size. 

\newpage % Page breaks are ignored

\section{Other formatting} 

Here is some {\em emphasized text}, {\bf boldface text}, 
{\tt typewriter font text}, and an \mbox{enclosed mbox}.  
You must ensure that these don't break across a line, {\em like
this}.  Footnotes don't appear at the bottom of the 
page\footnote{Instead footnotes are translated
inline, as you see here.}.  Footnotes are the only one of these
environments that can extend across lines.

\begin{quote}
Quotes are indented.
\end{quote}

% For now we don't do anything much with citations and cross-references.
% You can edit in the links by hand in the generated HTML.

% Paragraphs are not included in the table of contents
\paragraph{Cross references} The $TherapyControl$ schema in
section~\ref{sect:schema} is from~\cite{jacky95}.

\subsection*{Miscellaneous}

$TherapyControl$ or any other formal text at the beginning of a line
is handled properly.  \dots That's about all.
